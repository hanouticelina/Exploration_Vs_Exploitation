\part*{Introduction}
\addcontentsline{toc}{part}{Introduction}

En intelligence artificielle, plus pr\'ecis\'ement en \textit{Machine Learning}, on est souvent amené, pour un ensemble de choix possibles, \`a trouver un bon compromis entre l’exploration et l'exploitation. L'exploration consiste à obtenir d'avantage d'informations dans le but de prendre de meilleures décisions dans le futur. L’exploitation, quant \`a elle, consiste \`a agir de manière optimale en fonction de ce qui est d\'ej\`a connu. Ce dilemme constitue un probl\`eme largement \'etudi\'e dans le domaine de l’apprentissage par renforcement. En effet, beaucoup d'algorithmes ont \'et\'e propos\'es pour r\'esoudre un certain nombre de ces probl\`emes.

\bigskip
Dans ce projet, nous allons \'etudier et analyser un certain nombre d'algorithmes basés sur le dilemme de {\itshape l’exploration vs exploitation} \`a travers des IAs de jeu. Dans un premier temps, nous allons effectuer une évaluation expérimentale d'algorithmes classiques dans un cadre simplifi\'e. La deuxi\`eme partie est consacr\'ee \`a l'implémentation d'un algorithme de Monte Carlo et enfin, dans les deux dernières parties, nous allons \'etudier des algorithmes plus avanc\'es ainsi qu'un jeu de strat\'egie combinatoire. 

