\newpage
\part*{Conclusion}
\addcontentsline{toc}{part}{Conclusion}
Dans ce projet, nous avons mis en évidence le dilemme de l'exploitation $vs$ exploration à travers divers exemples, en l'occurrence deux jeux, où l'équilibre entre elles joue un rôle très important.

Au vu de nos expériences, nous pouvons confirmer que le paramètre modulant le choix entre les deux est dépendant du problème.

En ce qui concerne la machine à N leviers, nous avons constaté qu'un paramètre constant dans le temps peut être une solution acceptable, étant donné la simplicité de son implémentation, lorsque le besoin d'exploration est moindre.

La deuxième partie nous a servi pour vérifier expérimentalement une application de la loi de Bernoulli et son impact sur des comportements complexes dont elle est une brique de base.

Finalement, ce projet nous a permis de réaliser que, dans l'incertain, les estimations sont de la plus haute importance et c'est au moyen des probabilités que l'on détermine leur fiabilité.